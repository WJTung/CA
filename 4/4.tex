\documentclass[12pt, a4paper]{article}
\usepackage{fontspec}
\usepackage{xeCJK}
\usepackage{hyperref}
\usepackage{enumitem}
\setCJKmainfont{微軟正黑體}
\XeTeXlinebreaklocale "zh"
\XeTeXlinebreakskip = 0pt plus 1pt
\usepackage{enumerate}
\usepackage{graphicx}
\usepackage{color}
\usepackage{amsmath}
\date{}
\title{\vspace{-3.0cm} 計算機結構 \hspace{0cm} Exercise 04 \\ \vspace{0cm}}
\author{\normalsize B03902062 \hspace{0cm} 資工三 \hspace{0cm} 董文捷}
\begin{document}
\maketitle
\begin{itemize}[font=\bfseries]

\item[5.3.1] 5 bits are used for the offset, which means the cache block size is 32 byte = \textcolor{blue}{8} words

\item[5.3.2] 5 bits are used for the index, which means the cache have \textcolor{blue}{32} entries.

\item[5.3.3] $\displaystyle \frac{2^5 \times (2^5 \times 8 \makebox{ } (block\makebox{ }size) + 22 \makebox{ } (tag\makebox{ }size) + 1 \makebox{ } (valid\makebox{ }field\makebox{ }size))}{2^5 \times (2^5 \times 8)} = \textcolor{blue}{1.090}$

\item[5.3.4]
\begin{tabular}[t]{|c|c|c|c|} 
\hline
byte address & block address & block number & hit / miss \\
\hline
0 & $\lfloor 0 / 32 \rfloor$ = 0 & 0 mod 32 = 0 & miss \\
\hline
4 & $\lfloor 4 / 32 \rfloor$ = 0 & 0 mod 32 = 0 & hit \\
\hline
16 & $\lfloor 16 / 32 \rfloor$ = 0 & 0 mod 32 = 0 & hit \\
\hline
132 & $\lfloor 132 / 32 \rfloor$ = 4 & 4 mod 32 = 4 & miss \\
\hline
232 & $\lfloor 232 / 32 \rfloor$ = 7 & 7 mod 32 = 7 & miss \\
\hline
160 & $\lfloor 160 / 32 \rfloor$ = 5 & 5 mod 32 = 5 & miss \\
\hline
1024 & $\lfloor 1024 / 32 \rfloor$ = 32 & 32 mod 32 = 0 & miss (0 $\rightarrow$ 32) \\
\hline
30 & $\lfloor 30 / 32 \rfloor$ = 0 & 0 mod 32 = 0 & miss (32 $\rightarrow$ 0) \\
\hline
140 & $\lfloor 140 / 32 \rfloor$ = 4 & 4 mod 32 = 4 & hit \\
\hline
3100 & $\lfloor 3100 / 32 \rfloor$ = 96 & 96 mod 32 = 0 & miss (0 $\rightarrow$ 96)\\
\hline
180 & $\lfloor 180 / 32 \rfloor$ = 5 & 5 mod 32 = 5 & hit \\
\hline
2180 & $\lfloor 2180 / 32 \rfloor$ = 68 & 68 mod 32 = 4 & miss (4 $\rightarrow$ 68)\\
\hline
\end{tabular} \\
\vspace*{0.2cm} \\
\textcolor{blue}{4} replacement

\item[5.3.5] Hit ratio = $\displaystyle \frac{4}{12}$ = \textcolor{blue}{0.33}

\item[5.3.6]
<index, tag, data> \\
<$00000_2$, $0000\cdots0011_2$ (22 bits), mem[3072]> \\
<$00100_2$, $0000\cdots0010_2$ (22 bits), mem[2176]> \\
<$00101_2$, $0000\cdots0000_2$ (22 bits), mem[160]> \\
<$00111_2$, $0000\cdots0000_2$ (22 bits), mem[224]> 

\item[5.6.1]
\textbf{P1} : $\displaystyle \frac{1}{0.66 ns}$ = \textcolor{blue}{1.52} GHz \\
\textbf{P2} : $\displaystyle \frac{1}{0.90 ns}$ = \textcolor{blue}{1.11} GHz

\item[5.6.2]
\textbf{P1} : 0.66 ns + 8\% $\times$ 70 ns = \textcolor{blue}{6.26} ns \\
\textbf{P2} : 0.90 ns + 6\% $\times$ 70 ns = \textcolor{blue}{5.1} ns 

\item[5.6.3]
\textbf{P1} : $\displaystyle 1.0 + 36\% \times 8\% \times \frac{70}{0.66}$ = \textcolor{blue}{4.05} $\rightarrow$ 2.68 ns per instruction \\
\textbf{P2} : $\displaystyle 1.0 + 36\% \times 6\% \times \frac{70}{0.90}$ = \textcolor{blue}{2.68} $\rightarrow$ 2.41 ns per instruction \\ 
$\longrightarrow$ \textcolor{blue}{\textbf{P2}} is faster

\item[5.6.4]
AMAT = 0.66 ns + 8\% $\times$ 5.62 ns + 8\% $\times$ 95\% $\times$ 70 ns = \textcolor{blue}{6.43} ns \\ $\longrightarrow$ \textcolor{blue}{worse} 

\item[5.6.5]
$\displaystyle 1 + 36\% \times 8\% \times \frac{5.62}{0.66} + 36\% \times 8\% \times 95\% \times \frac{70}{0.66}$ = \textcolor{blue}{4.15}

\item[5.6.6]
\textbf{P1} : 2.74 ns per instruction \\
\textbf{P2} : 2.41 ns per instruction \\
$\longrightarrow$ \textcolor{blue}{\textbf{P2}} is faster \\
\vspace*{-0.3cm} \\
$\displaystyle 0.66 + 36\% \times \text{miss rate} \times 5.62 + 36\% \times \text{miss rate} \times 95\% \times 70$ \\ = $\displaystyle 0.90 + 36\% \times 6\% \times 70$ \\
$\longrightarrow$ \textbf{P1} needs miss rate in its L1 cache to be \textcolor{blue}{6.75\%} to match \textbf{P2}'s performance

\end{itemize}
\end{document}